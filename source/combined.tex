
% Default to the notebook output style

    


% Inherit from the specified cell style.




    
\documentclass[11pt]{article}

    
    
    \usepackage[T1]{fontenc}
    % Nicer default font (+ math font) than Computer Modern for most use cases
    \usepackage{mathpazo}

    % Basic figure setup, for now with no caption control since it's done
    % automatically by Pandoc (which extracts ![](path) syntax from Markdown).
    \usepackage{graphicx}
    % We will generate all images so they have a width \maxwidth. This means
    % that they will get their normal width if they fit onto the page, but
    % are scaled down if they would overflow the margins.
    \makeatletter
    \def\maxwidth{\ifdim\Gin@nat@width>\linewidth\linewidth
    \else\Gin@nat@width\fi}
    \makeatother
    \let\Oldincludegraphics\includegraphics
    % Set max figure width to be 80% of text width, for now hardcoded.
    \renewcommand{\includegraphics}[1]{\Oldincludegraphics[width=.8\maxwidth]{#1}}
    % Ensure that by default, figures have no caption (until we provide a
    % proper Figure object with a Caption API and a way to capture that
    % in the conversion process - todo).
    \usepackage{caption}
    \DeclareCaptionLabelFormat{nolabel}{}
    \captionsetup{labelformat=nolabel}

    \usepackage{adjustbox} % Used to constrain images to a maximum size 
    \usepackage{xcolor} % Allow colors to be defined
    \usepackage{enumerate} % Needed for markdown enumerations to work
    \usepackage{geometry} % Used to adjust the document margins
    \usepackage{amsmath} % Equations
    \usepackage{amssymb} % Equations
    \usepackage{textcomp} % defines textquotesingle
    % Hack from http://tex.stackexchange.com/a/47451/13684:
    \AtBeginDocument{%
        \def\PYZsq{\textquotesingle}% Upright quotes in Pygmentized code
    }
    \usepackage{upquote} % Upright quotes for verbatim code
    \usepackage{eurosym} % defines \euro
    \usepackage[mathletters]{ucs} % Extended unicode (utf-8) support
    \usepackage[utf8x]{inputenc} % Allow utf-8 characters in the tex document
    \usepackage{fancyvrb} % verbatim replacement that allows latex
    \usepackage{grffile} % extends the file name processing of package graphics 
                         % to support a larger range 
    % The hyperref package gives us a pdf with properly built
    % internal navigation ('pdf bookmarks' for the table of contents,
    % internal cross-reference links, web links for URLs, etc.)
    \usepackage{hyperref}
    \usepackage{longtable} % longtable support required by pandoc >1.10
    \usepackage{booktabs}  % table support for pandoc > 1.12.2
    \usepackage[inline]{enumitem} % IRkernel/repr support (it uses the enumerate* environment)
    \usepackage[normalem]{ulem} % ulem is needed to support strikethroughs (\sout)
                                % normalem makes italics be italics, not underlines
    

    
    
    % Colors for the hyperref package
    \definecolor{urlcolor}{rgb}{0,.145,.698}
    \definecolor{linkcolor}{rgb}{.71,0.21,0.01}
    \definecolor{citecolor}{rgb}{.12,.54,.11}

    % ANSI colors
    \definecolor{ansi-black}{HTML}{3E424D}
    \definecolor{ansi-black-intense}{HTML}{282C36}
    \definecolor{ansi-red}{HTML}{E75C58}
    \definecolor{ansi-red-intense}{HTML}{B22B31}
    \definecolor{ansi-green}{HTML}{00A250}
    \definecolor{ansi-green-intense}{HTML}{007427}
    \definecolor{ansi-yellow}{HTML}{DDB62B}
    \definecolor{ansi-yellow-intense}{HTML}{B27D12}
    \definecolor{ansi-blue}{HTML}{208FFB}
    \definecolor{ansi-blue-intense}{HTML}{0065CA}
    \definecolor{ansi-magenta}{HTML}{D160C4}
    \definecolor{ansi-magenta-intense}{HTML}{A03196}
    \definecolor{ansi-cyan}{HTML}{60C6C8}
    \definecolor{ansi-cyan-intense}{HTML}{258F8F}
    \definecolor{ansi-white}{HTML}{C5C1B4}
    \definecolor{ansi-white-intense}{HTML}{A1A6B2}

    % commands and environments needed by pandoc snippets
    % extracted from the output of `pandoc -s`
    \providecommand{\tightlist}{%
      \setlength{\itemsep}{0pt}\setlength{\parskip}{0pt}}
    \DefineVerbatimEnvironment{Highlighting}{Verbatim}{commandchars=\\\{\}}
    % Add ',fontsize=\small' for more characters per line
    \newenvironment{Shaded}{}{}
    \newcommand{\KeywordTok}[1]{\textcolor[rgb]{0.00,0.44,0.13}{\textbf{{#1}}}}
    \newcommand{\DataTypeTok}[1]{\textcolor[rgb]{0.56,0.13,0.00}{{#1}}}
    \newcommand{\DecValTok}[1]{\textcolor[rgb]{0.25,0.63,0.44}{{#1}}}
    \newcommand{\BaseNTok}[1]{\textcolor[rgb]{0.25,0.63,0.44}{{#1}}}
    \newcommand{\FloatTok}[1]{\textcolor[rgb]{0.25,0.63,0.44}{{#1}}}
    \newcommand{\CharTok}[1]{\textcolor[rgb]{0.25,0.44,0.63}{{#1}}}
    \newcommand{\StringTok}[1]{\textcolor[rgb]{0.25,0.44,0.63}{{#1}}}
    \newcommand{\CommentTok}[1]{\textcolor[rgb]{0.38,0.63,0.69}{\textit{{#1}}}}
    \newcommand{\OtherTok}[1]{\textcolor[rgb]{0.00,0.44,0.13}{{#1}}}
    \newcommand{\AlertTok}[1]{\textcolor[rgb]{1.00,0.00,0.00}{\textbf{{#1}}}}
    \newcommand{\FunctionTok}[1]{\textcolor[rgb]{0.02,0.16,0.49}{{#1}}}
    \newcommand{\RegionMarkerTok}[1]{{#1}}
    \newcommand{\ErrorTok}[1]{\textcolor[rgb]{1.00,0.00,0.00}{\textbf{{#1}}}}
    \newcommand{\NormalTok}[1]{{#1}}
    
    % Additional commands for more recent versions of Pandoc
    \newcommand{\ConstantTok}[1]{\textcolor[rgb]{0.53,0.00,0.00}{{#1}}}
    \newcommand{\SpecialCharTok}[1]{\textcolor[rgb]{0.25,0.44,0.63}{{#1}}}
    \newcommand{\VerbatimStringTok}[1]{\textcolor[rgb]{0.25,0.44,0.63}{{#1}}}
    \newcommand{\SpecialStringTok}[1]{\textcolor[rgb]{0.73,0.40,0.53}{{#1}}}
    \newcommand{\ImportTok}[1]{{#1}}
    \newcommand{\DocumentationTok}[1]{\textcolor[rgb]{0.73,0.13,0.13}{\textit{{#1}}}}
    \newcommand{\AnnotationTok}[1]{\textcolor[rgb]{0.38,0.63,0.69}{\textbf{\textit{{#1}}}}}
    \newcommand{\CommentVarTok}[1]{\textcolor[rgb]{0.38,0.63,0.69}{\textbf{\textit{{#1}}}}}
    \newcommand{\VariableTok}[1]{\textcolor[rgb]{0.10,0.09,0.49}{{#1}}}
    \newcommand{\ControlFlowTok}[1]{\textcolor[rgb]{0.00,0.44,0.13}{\textbf{{#1}}}}
    \newcommand{\OperatorTok}[1]{\textcolor[rgb]{0.40,0.40,0.40}{{#1}}}
    \newcommand{\BuiltInTok}[1]{{#1}}
    \newcommand{\ExtensionTok}[1]{{#1}}
    \newcommand{\PreprocessorTok}[1]{\textcolor[rgb]{0.74,0.48,0.00}{{#1}}}
    \newcommand{\AttributeTok}[1]{\textcolor[rgb]{0.49,0.56,0.16}{{#1}}}
    \newcommand{\InformationTok}[1]{\textcolor[rgb]{0.38,0.63,0.69}{\textbf{\textit{{#1}}}}}
    \newcommand{\WarningTok}[1]{\textcolor[rgb]{0.38,0.63,0.69}{\textbf{\textit{{#1}}}}}
    
    
    % Define a nice break command that doesn't care if a line doesn't already
    % exist.
    \def\br{\hspace*{\fill} \\* }
    % Math Jax compatability definitions
    \def\gt{>}
    \def\lt{<}
    % Document parameters
    \title{Notebook}
    
    
    

    % Pygments definitions
    
\makeatletter
\def\PY@reset{\let\PY@it=\relax \let\PY@bf=\relax%
    \let\PY@ul=\relax \let\PY@tc=\relax%
    \let\PY@bc=\relax \let\PY@ff=\relax}
\def\PY@tok#1{\csname PY@tok@#1\endcsname}
\def\PY@toks#1+{\ifx\relax#1\empty\else%
    \PY@tok{#1}\expandafter\PY@toks\fi}
\def\PY@do#1{\PY@bc{\PY@tc{\PY@ul{%
    \PY@it{\PY@bf{\PY@ff{#1}}}}}}}
\def\PY#1#2{\PY@reset\PY@toks#1+\relax+\PY@do{#2}}

\expandafter\def\csname PY@tok@w\endcsname{\def\PY@tc##1{\textcolor[rgb]{0.73,0.73,0.73}{##1}}}
\expandafter\def\csname PY@tok@c\endcsname{\let\PY@it=\textit\def\PY@tc##1{\textcolor[rgb]{0.25,0.50,0.50}{##1}}}
\expandafter\def\csname PY@tok@cp\endcsname{\def\PY@tc##1{\textcolor[rgb]{0.74,0.48,0.00}{##1}}}
\expandafter\def\csname PY@tok@k\endcsname{\let\PY@bf=\textbf\def\PY@tc##1{\textcolor[rgb]{0.00,0.50,0.00}{##1}}}
\expandafter\def\csname PY@tok@kp\endcsname{\def\PY@tc##1{\textcolor[rgb]{0.00,0.50,0.00}{##1}}}
\expandafter\def\csname PY@tok@kt\endcsname{\def\PY@tc##1{\textcolor[rgb]{0.69,0.00,0.25}{##1}}}
\expandafter\def\csname PY@tok@o\endcsname{\def\PY@tc##1{\textcolor[rgb]{0.40,0.40,0.40}{##1}}}
\expandafter\def\csname PY@tok@ow\endcsname{\let\PY@bf=\textbf\def\PY@tc##1{\textcolor[rgb]{0.67,0.13,1.00}{##1}}}
\expandafter\def\csname PY@tok@nb\endcsname{\def\PY@tc##1{\textcolor[rgb]{0.00,0.50,0.00}{##1}}}
\expandafter\def\csname PY@tok@nf\endcsname{\def\PY@tc##1{\textcolor[rgb]{0.00,0.00,1.00}{##1}}}
\expandafter\def\csname PY@tok@nc\endcsname{\let\PY@bf=\textbf\def\PY@tc##1{\textcolor[rgb]{0.00,0.00,1.00}{##1}}}
\expandafter\def\csname PY@tok@nn\endcsname{\let\PY@bf=\textbf\def\PY@tc##1{\textcolor[rgb]{0.00,0.00,1.00}{##1}}}
\expandafter\def\csname PY@tok@ne\endcsname{\let\PY@bf=\textbf\def\PY@tc##1{\textcolor[rgb]{0.82,0.25,0.23}{##1}}}
\expandafter\def\csname PY@tok@nv\endcsname{\def\PY@tc##1{\textcolor[rgb]{0.10,0.09,0.49}{##1}}}
\expandafter\def\csname PY@tok@no\endcsname{\def\PY@tc##1{\textcolor[rgb]{0.53,0.00,0.00}{##1}}}
\expandafter\def\csname PY@tok@nl\endcsname{\def\PY@tc##1{\textcolor[rgb]{0.63,0.63,0.00}{##1}}}
\expandafter\def\csname PY@tok@ni\endcsname{\let\PY@bf=\textbf\def\PY@tc##1{\textcolor[rgb]{0.60,0.60,0.60}{##1}}}
\expandafter\def\csname PY@tok@na\endcsname{\def\PY@tc##1{\textcolor[rgb]{0.49,0.56,0.16}{##1}}}
\expandafter\def\csname PY@tok@nt\endcsname{\let\PY@bf=\textbf\def\PY@tc##1{\textcolor[rgb]{0.00,0.50,0.00}{##1}}}
\expandafter\def\csname PY@tok@nd\endcsname{\def\PY@tc##1{\textcolor[rgb]{0.67,0.13,1.00}{##1}}}
\expandafter\def\csname PY@tok@s\endcsname{\def\PY@tc##1{\textcolor[rgb]{0.73,0.13,0.13}{##1}}}
\expandafter\def\csname PY@tok@sd\endcsname{\let\PY@it=\textit\def\PY@tc##1{\textcolor[rgb]{0.73,0.13,0.13}{##1}}}
\expandafter\def\csname PY@tok@si\endcsname{\let\PY@bf=\textbf\def\PY@tc##1{\textcolor[rgb]{0.73,0.40,0.53}{##1}}}
\expandafter\def\csname PY@tok@se\endcsname{\let\PY@bf=\textbf\def\PY@tc##1{\textcolor[rgb]{0.73,0.40,0.13}{##1}}}
\expandafter\def\csname PY@tok@sr\endcsname{\def\PY@tc##1{\textcolor[rgb]{0.73,0.40,0.53}{##1}}}
\expandafter\def\csname PY@tok@ss\endcsname{\def\PY@tc##1{\textcolor[rgb]{0.10,0.09,0.49}{##1}}}
\expandafter\def\csname PY@tok@sx\endcsname{\def\PY@tc##1{\textcolor[rgb]{0.00,0.50,0.00}{##1}}}
\expandafter\def\csname PY@tok@m\endcsname{\def\PY@tc##1{\textcolor[rgb]{0.40,0.40,0.40}{##1}}}
\expandafter\def\csname PY@tok@gh\endcsname{\let\PY@bf=\textbf\def\PY@tc##1{\textcolor[rgb]{0.00,0.00,0.50}{##1}}}
\expandafter\def\csname PY@tok@gu\endcsname{\let\PY@bf=\textbf\def\PY@tc##1{\textcolor[rgb]{0.50,0.00,0.50}{##1}}}
\expandafter\def\csname PY@tok@gd\endcsname{\def\PY@tc##1{\textcolor[rgb]{0.63,0.00,0.00}{##1}}}
\expandafter\def\csname PY@tok@gi\endcsname{\def\PY@tc##1{\textcolor[rgb]{0.00,0.63,0.00}{##1}}}
\expandafter\def\csname PY@tok@gr\endcsname{\def\PY@tc##1{\textcolor[rgb]{1.00,0.00,0.00}{##1}}}
\expandafter\def\csname PY@tok@ge\endcsname{\let\PY@it=\textit}
\expandafter\def\csname PY@tok@gs\endcsname{\let\PY@bf=\textbf}
\expandafter\def\csname PY@tok@gp\endcsname{\let\PY@bf=\textbf\def\PY@tc##1{\textcolor[rgb]{0.00,0.00,0.50}{##1}}}
\expandafter\def\csname PY@tok@go\endcsname{\def\PY@tc##1{\textcolor[rgb]{0.53,0.53,0.53}{##1}}}
\expandafter\def\csname PY@tok@gt\endcsname{\def\PY@tc##1{\textcolor[rgb]{0.00,0.27,0.87}{##1}}}
\expandafter\def\csname PY@tok@err\endcsname{\def\PY@bc##1{\setlength{\fboxsep}{0pt}\fcolorbox[rgb]{1.00,0.00,0.00}{1,1,1}{\strut ##1}}}
\expandafter\def\csname PY@tok@kc\endcsname{\let\PY@bf=\textbf\def\PY@tc##1{\textcolor[rgb]{0.00,0.50,0.00}{##1}}}
\expandafter\def\csname PY@tok@kd\endcsname{\let\PY@bf=\textbf\def\PY@tc##1{\textcolor[rgb]{0.00,0.50,0.00}{##1}}}
\expandafter\def\csname PY@tok@kn\endcsname{\let\PY@bf=\textbf\def\PY@tc##1{\textcolor[rgb]{0.00,0.50,0.00}{##1}}}
\expandafter\def\csname PY@tok@kr\endcsname{\let\PY@bf=\textbf\def\PY@tc##1{\textcolor[rgb]{0.00,0.50,0.00}{##1}}}
\expandafter\def\csname PY@tok@bp\endcsname{\def\PY@tc##1{\textcolor[rgb]{0.00,0.50,0.00}{##1}}}
\expandafter\def\csname PY@tok@fm\endcsname{\def\PY@tc##1{\textcolor[rgb]{0.00,0.00,1.00}{##1}}}
\expandafter\def\csname PY@tok@vc\endcsname{\def\PY@tc##1{\textcolor[rgb]{0.10,0.09,0.49}{##1}}}
\expandafter\def\csname PY@tok@vg\endcsname{\def\PY@tc##1{\textcolor[rgb]{0.10,0.09,0.49}{##1}}}
\expandafter\def\csname PY@tok@vi\endcsname{\def\PY@tc##1{\textcolor[rgb]{0.10,0.09,0.49}{##1}}}
\expandafter\def\csname PY@tok@vm\endcsname{\def\PY@tc##1{\textcolor[rgb]{0.10,0.09,0.49}{##1}}}
\expandafter\def\csname PY@tok@sa\endcsname{\def\PY@tc##1{\textcolor[rgb]{0.73,0.13,0.13}{##1}}}
\expandafter\def\csname PY@tok@sb\endcsname{\def\PY@tc##1{\textcolor[rgb]{0.73,0.13,0.13}{##1}}}
\expandafter\def\csname PY@tok@sc\endcsname{\def\PY@tc##1{\textcolor[rgb]{0.73,0.13,0.13}{##1}}}
\expandafter\def\csname PY@tok@dl\endcsname{\def\PY@tc##1{\textcolor[rgb]{0.73,0.13,0.13}{##1}}}
\expandafter\def\csname PY@tok@s2\endcsname{\def\PY@tc##1{\textcolor[rgb]{0.73,0.13,0.13}{##1}}}
\expandafter\def\csname PY@tok@sh\endcsname{\def\PY@tc##1{\textcolor[rgb]{0.73,0.13,0.13}{##1}}}
\expandafter\def\csname PY@tok@s1\endcsname{\def\PY@tc##1{\textcolor[rgb]{0.73,0.13,0.13}{##1}}}
\expandafter\def\csname PY@tok@mb\endcsname{\def\PY@tc##1{\textcolor[rgb]{0.40,0.40,0.40}{##1}}}
\expandafter\def\csname PY@tok@mf\endcsname{\def\PY@tc##1{\textcolor[rgb]{0.40,0.40,0.40}{##1}}}
\expandafter\def\csname PY@tok@mh\endcsname{\def\PY@tc##1{\textcolor[rgb]{0.40,0.40,0.40}{##1}}}
\expandafter\def\csname PY@tok@mi\endcsname{\def\PY@tc##1{\textcolor[rgb]{0.40,0.40,0.40}{##1}}}
\expandafter\def\csname PY@tok@il\endcsname{\def\PY@tc##1{\textcolor[rgb]{0.40,0.40,0.40}{##1}}}
\expandafter\def\csname PY@tok@mo\endcsname{\def\PY@tc##1{\textcolor[rgb]{0.40,0.40,0.40}{##1}}}
\expandafter\def\csname PY@tok@ch\endcsname{\let\PY@it=\textit\def\PY@tc##1{\textcolor[rgb]{0.25,0.50,0.50}{##1}}}
\expandafter\def\csname PY@tok@cm\endcsname{\let\PY@it=\textit\def\PY@tc##1{\textcolor[rgb]{0.25,0.50,0.50}{##1}}}
\expandafter\def\csname PY@tok@cpf\endcsname{\let\PY@it=\textit\def\PY@tc##1{\textcolor[rgb]{0.25,0.50,0.50}{##1}}}
\expandafter\def\csname PY@tok@c1\endcsname{\let\PY@it=\textit\def\PY@tc##1{\textcolor[rgb]{0.25,0.50,0.50}{##1}}}
\expandafter\def\csname PY@tok@cs\endcsname{\let\PY@it=\textit\def\PY@tc##1{\textcolor[rgb]{0.25,0.50,0.50}{##1}}}

\def\PYZbs{\char`\\}
\def\PYZus{\char`\_}
\def\PYZob{\char`\{}
\def\PYZcb{\char`\}}
\def\PYZca{\char`\^}
\def\PYZam{\char`\&}
\def\PYZlt{\char`\<}
\def\PYZgt{\char`\>}
\def\PYZsh{\char`\#}
\def\PYZpc{\char`\%}
\def\PYZdl{\char`\$}
\def\PYZhy{\char`\-}
\def\PYZsq{\char`\'}
\def\PYZdq{\char`\"}
\def\PYZti{\char`\~}
% for compatibility with earlier versions
\def\PYZat{@}
\def\PYZlb{[}
\def\PYZrb{]}
\makeatother


    % Exact colors from NB
    \definecolor{incolor}{rgb}{0.0, 0.0, 0.5}
    \definecolor{outcolor}{rgb}{0.545, 0.0, 0.0}



    
    % Prevent overflowing lines due to hard-to-break entities
    \sloppy 
    % Setup hyperref package
    \hypersetup{
      breaklinks=true,  % so long urls are correctly broken across lines
      colorlinks=true,
      urlcolor=urlcolor,
      linkcolor=linkcolor,
      citecolor=citecolor,
      }
    % Slightly bigger margins than the latex defaults
    
    \geometry{verbose,tmargin=1in,bmargin=1in,lmargin=1in,rmargin=1in}
    
    

    \begin{document}
    
    
    \maketitle
    
    

    
    \section{LMTH 2040: Multidisciplinary
Calculus}\label{lmth-2040-multidisciplinary-calculus}
\label{sec:01-syllabus}
    \begin{figure}
\centering
\includegraphics{images/section_I/logo.jpg}
\caption{}
\end{figure}

\textbf{Spring 2018}

\textbf{Monday and Wednesday 12:00 p.m - 1:15 p.m.}

\textbf{261 65 West 11th St.}

\textbf{Email: koehlerj@newschool.edu}

Office Hours: by appointment

\begin{center}\rule{0.5\linewidth}{\linethickness}\end{center}

\subsubsection{Overview}\label{overview}

This class is an introduction to basic ideas and applications of the
Calculus. The content focuses on integration, differentiation, and
differential equations and how these can be used in several real
applications. We will use the computer consistently, introducing the
Python computer language to complete our work.

\subsubsection{Learning Objectives}\label{learning-objectives}

\begin{itemize}
\tightlist
\item
  Understand the mathematical concepts and history of Integration
\item
  Use integration to solve problems
\item
  Understand the mathematical concept and history of Differentiation
\item
  Use derivatives and differentiation to solve problems
\item
  Connect Integration and Differentiation through Differential Equations
  and explore the history of these problems
\item
  Use differential equations to solve problems
\end{itemize}

\subsubsection{Course Requirements}\label{course-requirements}

\begin{itemize}
\tightlist
\item
  Complete Weekly Problem Sets
\item
  Ask and answer questions on Piazza or Stack Exchange
\item
  Complete 6 computer labs
\item
  Complete three independent investigations
\item
  Complete Final Project or Problem Set
\end{itemize}

\subsubsection{Final Grade Calculuation}\label{final-grade-calculuation}

\begin{itemize}
\tightlist
\item
  Participation/Attendance 20\%
\item
  Group Problem Sets 20\%
\item
  Individual Problem Sets 20\%
\item
  Computer Labs 20\%
\item
  Final 20\%
\end{itemize}

\subsubsection{Course Policies}\label{course-policies}

We will rely heavily on peers and open collaboration on our assignments.
We will use a variety of resources for the class, described below:

\begin{itemize}
\tightlist
\item
  Assignments and announcements will be posted on our Canvas Page
\item
  Course Materials from proprietary sources will also be shared through
  Canvas
\item
  Course Website: http://spring-2018-calc.readthedocs.io/en/latest/
\item
  Students are free to ask and post responses in our Piazza discussion
  board, available here:
  \url{https://piazza.com/newschool/spring2018/lmth2040/home}.
\item
  Students are free to ask questions on open message boards;
  StackExchange in particular. Please repost any questions like this in
  our Piazza page.
\end{itemize}

\subsubsection{Course Reading Materials}\label{course-reading-materials}

We will use two freely available textbooks this semester. Make sure you
can locate and access them:

\begin{itemize}
\tightlist
\item
  OpenStax Calculus Textbook
  (\href{https://openstax.org/details/books/calculus-volume-1}{pdf
  here})
\item
  The Origins of Calculus by David Perkins (ebook in library)
\end{itemize}

There are many many other freely accessible resources for learning
calculus such as EdX, Coursera, Khan Academy, etc. Feel free to make
ample use of these through the semester.

\subsubsection{Resources}\label{resources}

\textbf{IS Student Support}

Rachel Gottlieb, B.A. Candidate in Interdisciplinary Science and Gender
Studies gottr694@gmail.com Availability: Mondays \& Wednesdays after
3:30 in Room 459 Science Lab, 65 West 11th Street

Marina Delgado, B.A. Candidate in Interdisciplinary Science and Poetry
Email: delgm708@newschool.edu Mondays and Wednesdays: 3:30 PM - 5:30 PM
(except for Job Talk dates) Tuesdays: 12:30 PM - 3:00 PM, Room 459
Science Lab, 65 West 11th Street Fridays are flexible!

The university provides many resources to help students achieve academic
and artistic excellence. These resources include:

\begin{itemize}
\tightlist
\item
  The University (and associated) Libraries:
  http://library.newschool.edu
\item
  The University Learning Center:
  http://www.newschool.edu/learning-center
\item
  University Disabilities Service:
  www.newschool.edu/student-disability-services/ In keeping with the
  university's policy of providing equal access for students with
  disabilities, any student with a disability who needs academic
  accommodations is welcome to meet with me privately. All conversations
  will be kept confidential. Students requesting any accommodations will
  also need to contact Student Disability Service (SDS). SDS will
  conduct an intake and, if appropriate, the Director will provide an
  academic accommodation notification letter for you to bring to me. At
  that point, I will review the letter with you and discuss these
  accommodations in relation to this course.
\end{itemize}

\subsubsection{Schedule}\label{schedule}

\begin{itemize}
\item
  Monday, January 22. Introduction to Class: What is a number?
  Investigation integers, rational numbers, irrational numbers, and
  related historical problems.
\item
  Wednesday, January 24. Number and Algorithm: Continue to work on
  problem set in groups. Problem write-up workshop.
\item
  \textbf{Computer Lab I} Monday, January 29: Sequences, Functions,
  Summations with Python
\item
  Wednesday, January 31. Summations: Interpret problems with areas. Add
  rectangles, trapezoids, and parabolas, discuss accuracy.
\item
  Monday, February 5. Definite Integral: Introduce definition of
  definite integral, basic rules for polynomial, trigonometric, and
  exponential functions. Use technology and tables.
\item
  ** Computer Lab II** Wednesday, February 7: Discrete and Continuous
  Distributions with Python.
\item
  Monday and Wednesday, February 12 - 14.
\end{itemize}

\textbf{PROJECT I}: Statistics of distributions, Voting Power,
Work-Force-Center of Mass, Numerical Algorithms for Integration, History
of Integration, Consumer/Producer Surplus

\begin{center}\rule{0.5\linewidth}{\linethickness}\end{center}

\begin{itemize}
\item
  Wednesday, February 21. Differences and Derivatives: Investigate the
  discrete case of first and second differences, connect these with
  average rates of change.
\item
  Monday, February 26. Slopes and a Definition: Use slopes to move to
  continuous case, discuss approximate and exact solutions, connect
  first and second derivatives with first and second differences.
\item
  Wednesday, February 28. Lab III. Derivatives with Python.
\item
  Monday March 5. Application of Derivatives I: Historical Problems
  related to Derivatives.
\item
  Wednesday, March 7. Application of Derivatives II: Vectors,
  Trigonometry, and Problems from Physics.
\item
  Monday, March 12. Lab IV. Optimization in Python.
\item
  Wednesday, March 14. Differentiation Review: Various problems
  involving differentiation.
\item
  Week of March 19 and 21.
\end{itemize}

\textbf{PROJECT II}: Linear Regression, Arbitrating Disputes, Item
Response Ideas, Motion of Projectiles, Numerical Differentiation,
History of Differentiation, Elasticity of Demand

\begin{center}\rule{0.5\linewidth}{\linethickness}\end{center}

\begin{itemize}
\item
  Monday, March 26. Recursion: Using the logistic population, focus on
  models built recursively using rates of change.
\item
  Wednesday, March 28. Differential Equations: Antiderivates and solving
  separable differential equations.
\item
  Monday, April 2. Lab V - Modeling Differential Equations with Python
\item
  Wednesday, April 4. Systems: Model population change in time with
  Lotka-Volterra
\item
  Monday, April 9. Qualitative Analysis of ODE's: Phase Plane Analysis
\item
  Wednesday, April 11. Numerical Approaches: Euler and Runge-Kutta
\item
  Monday, April 16. Lab VI: Solving and Visualizing ODE's with Python.
\item
  Wednesday, April 18. Differential Equations Review.
\item
  Week of April 23 - 25.
\end{itemize}

\textbf{PROJECT III}: Gradient Descent, Dynamical Systems and Social
Theory, Differential Equations and Physics, CobWeb Diagrams in
Economics, Advanced Numerical Solutions, History of Differential
Equations.

\begin{center}\rule{0.5\linewidth}{\linethickness}\end{center}

\begin{itemize}
\tightlist
\item
  April 20 - May 9
\end{itemize}

\textbf{FINAL PROJECT}: Paper or Problem Set.

    \subsubsection{Technology}\label{technology}

We wil use Python and Jupyter Notebooks to complete our work. You should
download and install these through the Anaconda program, freely
available \href{https://www.anaconda.com/download/\#macos}{here}.

    \section{Problem Set I}\label{problem-set-i}
\label{sec:02-problemsetI}
    \textbf{Mathematical Goals}:

\begin{itemize}
\tightlist
\item
  Solve problems related to areas
\item
  Understand Base number notation
\item
  Perform basic operations on numbers including root extraction
\item
  Use Pythagorean Theorem to find distances
\item
  Solve quadratic equations
\item
  Investigate partial sums of series
\end{itemize}

The problems below come from the marvelous text \emph{The Historical
Development of Calculus} by C.H. Edwards.

\textbf{Suggested Reading}:

\begin{itemize}
\tightlist
\item
  Chapter 1: \emph{The Historical Development of Calculus}, C.H.Edwards
\item
  Chapter 1: \emph{The Origins of Calculus}, David Perkins
\item
  \emph{History and Origins of the Calculus}, G.W. Leibniz
\end{itemize}

    \subsubsection{Problem I}\label{problem-i}

\begin{figure}
\centering
\includegraphics{images/section_I/p1.png}
\caption{}
\end{figure}

Use the images above to explain the formulas for the areas:

\begin{itemize}
\tightlist
\item
  of triangles \(A = \frac{1}{2}bh\)
\item
  of parallelograms \(A = bh\)
\item
  of trapezoids \(A = \frac{1}{2}(b_1 + b_2)h\)
\end{itemize}

    \subsubsection{Problem II}\label{problem-ii}

An Egyptian Papyrus calculates the area of a quadrilateral by
multiplying half the sum of two opposite sides times half the sum of the
other two sides. Is this the correct result for a trapezoid or
parallelogram that is not a rectangle?

    \subsubsection{Problem III}\label{problem-iii}

\begin{enumerate}
\def\labelenumi{\alph{enumi}.}
\tightlist
\item
  In one of the Rhind papyrus problems the area of a cirle is calculated
  by squaring 8/9 of its diameter. Compare this method with the area
  formula \(A = \pi r^2\) to obtain the Egyptian approximation of
  \(\pi = 3.16\).
\item
  This approximation to \(\pi\) may have been found by trisecting each
  side of a square circumscribed about a circle of diameter \(d\), and
  cut off its 4 corners. Show that the area here would be:
\end{enumerate}

\[ A = \frac{7}{9}d^2 = \frac{63}{81}d^2 \approx \frac{64}{81}d^2 = (\frac{8}{9}d)^2\]

\begin{figure}
\centering
\includegraphics{images/section_I/p2.png}
\caption{}
\end{figure}

    \subsubsection{Problem IV}\label{problem-iv}

Four copies of a right triangle with legs \(a\) and \(b\) and hypotenuse
\(c\) together with a square of edge \(c\), are assembled as in Figure 3
to form a square of edge \(a + b\). Explain why the assembled figure
\emph{is} a squre, and derive the Pythagorean relation by computing its
area in two different ways.

\begin{figure}
\centering
\includegraphics{images/section_I/p3.png}
\caption{}
\end{figure}

    \subsubsection{Problem V}\label{problem-v}

The Babylonians generally used \(3r^2\) for the area of a circle of
radius \(r\), corresponding to the poor approximation \(\pi \approx 3\).
Show that this approxiamation could have been obtained by averaging the
areas of the iscribed and circumscribed squares shown below.

\begin{figure}
\centering
\includegraphics{images/section_I/p4.png}
\caption{}
\end{figure}

    \subsubsection{Problem VI}\label{problem-vi}

Archimedes took a similar approach to approximating \(\pi\). He began by
inscribing and circumscribing a circle with regular hexagons, and
successively doubled the sides in order to, calculuating their
perimeters to find upper and lower bounds for \(\pi\). Beginning with a
circle of radius 1 and compute the perimeter for the following shapes:

\begin{longtable}[]{@{}ll@{}}
\toprule
Polygon & Perimeter\tabularnewline
\midrule
\endhead
Inscribed 6 sided &\tabularnewline
Circumscribed 6 sided &\tabularnewline
In 12 Sided &\tabularnewline
Circum 12 Sided &\tabularnewline
\bottomrule
\end{longtable}

    \subsubsection{Problem VII}\label{problem-vii}

The Babylonians approached square root approximation in a similar
iterative methodology like Archimedes use of the method of exhaustion.
To start, suppose we have a guess that we think is close to \(\sqrt{2}\)

\[x_1 \approx \sqrt{2} \quad \rightarrow \quad x_1 \times x_1 \approx 2 \quad \rightarrow \quad x_1 \approx \frac{2}{x_1}\]

Either \(x_1\) is a better guess or \(\frac{2}{x}\), but even better
still would be the average of the two:

\[x_2 = \frac{1}{2} \big(x_1 + \frac{2}{x_1}\big)\]

If we continue in this manner we will get better and better
approximations:

\[x_3 = \frac{1}{2} \big(x_2 + \frac{2}{x_2}\big)\]

\[x_4 = \frac{1}{2} \big(x_3 + \frac{2}{x_3}\big)\]

\[x_5 = \frac{1}{2} \big(x_4 + \frac{2}{x_4}\big)\]

\[\vdots\]

\[x_{n+1} = \frac{1}{2} \big(x_n + \frac{2}{x_n}\big)\]

Follow the Babylonians method to approximate \(\sqrt{2}\) through
\(x_5\). Repeat for \(\sqrt{3}\).

    \subsubsection{Problem VIII}\label{problem-viii}

The Arabic mathematician Al - Khowarizmi's introduced the base 10
numeration system to popular audiences in his writing. Here, collections
of groups of 10 items were combined with positional notation and zero to
make our familiar numeration system. For example, we would write 13,285
as

\[13285 = 1*10^4 + 3*10^3 + 2*10^2 + 8*10^1 + 5*10^0\]

Consider groupings different than base 10 for the following questions:

\begin{enumerate}
\def\labelenumi{\arabic{enumi}.}
\tightlist
\item
  How many numerals are required for a base 5 numeration system?
\item
  Can you express the base 10 numbers 360, 78, 35, and 23 in base 5
  notation? (Ex 10 base 10 = \(20_5\)(base 5))
\item
  0.3012 stands for \[3/10 + 0/10^2 + 1/10^3 + 2/10^4\]. Write the same
  number as a decimal in base 5.
\end{enumerate}

    \subsubsection{Problem IX}\label{problem-ix}

Al - Khowarizimi also discussed an approach to solving quadratic
equations such as \(x^2 + 10x = 39\). According to Edwards, his solution
was

\begin{itemize}
\tightlist
\item
  Take half the number of roots, that is, five, and multply this by
  itself to obtain twenty-five.\\
\item
  Add this to the thiry-nine, giving sixty-four.\\
\item
  Take the square root, or eight, and subtract from it half the number
  of roots (five).
\item
  The result, 3, is the required root.
\end{itemize}

\begin{figure}
\centering
\includegraphics{images/section_I/p5.png}
\caption{}
\end{figure}

Solve the equation \(x^2 + 8x = 65\) following a similar construction to
that above.

    \subsubsection{Problem X}\label{problem-x}

The images below help to understand the value of expressions like:

\[ 1 + 2 + 3 + 4 + ... + (n-1) + n = \]

\begin{figure}
\centering
\includegraphics{images/section_I/p6.png}
\caption{}
\end{figure}

\begin{enumerate}
\def\labelenumi{\arabic{enumi}.}
\tightlist
\item
  Find the sum when \(n = 5\)
\item
  Find the sum when \(n = 10\)
\item
  Is there a way to use the image below to understand the sum when
  \(n = 100\)? \includegraphics{images/section_I/p7.png}
\end{enumerate}

    \subsubsection{Problem XI}\label{problem-xi}

In the seventeenth century, Leibniz introduced modern summation
notation. The previous problems sum would be written as

\[\sum_{i = 1}^n i \]

which indicated we are adding (\(\sum\)), starting with an index of 1
(\(i = 1\)), and the things we are adding are simply the indices
(\(i\)). Sometimes, we recognize patterns in the partial sums of terms
as a way to understand the general approach. Consider the summation

\[\sum_{i = 1}^n i^3\]

Fill in the table with the following partial sums. Do you see a pattern?
Describe it.

\begin{longtable}[]{@{}ll@{}}
\toprule
\(n\) & Partial Sum \(\sum_{i = 1}^n i^3\)\tabularnewline
\midrule
\endhead
1 &\tabularnewline
2 &\tabularnewline
3 &\tabularnewline
4 &\tabularnewline
10 &\tabularnewline
j &\tabularnewline
\bottomrule
\end{longtable}

    \section{Rubric for Problem Sets}\label{rubric-for-problem-sets}
\label{sec:03-solutionsrubric}
    We will focus on four areas to determine the quality of solutions in our
work. First and most important is that you are writing your solutions to
a reader, and you should be considerate of using devices that help the
reader understand your solutions. The four components follow:

\begin{itemize}
\tightlist
\item
  Clearly State the Problem in your own words
\item
  Describe the approach to the solution
\item
  Show solution method using multiple representations: words, tables,
  graphs/pictures, formulas, numbers
\item
  Summarize and discuss
\end{itemize}

\subsubsection{Goals Today}\label{goals-today}

\begin{enumerate}
\def\labelenumi{\arabic{enumi}.}
\tightlist
\item
  Establish Groups
\item
  Finalize Tech
\item
  Problems V - VII
\item
  Feedback - Questions - Student Information
\item
  Homework:
\end{enumerate}

\begin{itemize}
\tightlist
\item
  Problem Set: Complete and write up, submit single group assignment
\item
  Piazza: Post Information and StackExchange Handle
\item
  Anaconda and Jupyter: Download and Open
\end{itemize}

\begin{enumerate}
\def\labelenumi{\arabic{enumi}.}
\setcounter{enumi}{5}
\tightlist
\item
  Next Class:
\end{enumerate}

\begin{itemize}
\tightlist
\item
  Introduction to Python and Jupyter
\end{itemize}


    % Add a bibliography block to the postdoc
    
    
    
    \end{document}
